Let Step a generate independent (U_1,U_2 \sim \mathrm{U}[-1,1]) until (Y = U_1^2 + U_2^2 \le 1). Define (\vartheta = \operatorname{atan2}(U_1,U_2)), and in Step b set
[
Z = \sqrt{-2\log Y}, \qquad
X_1 = Z,\frac{U_1}{\sqrt{Y}}, \quad X_2 = Z,\frac{U_2}{\sqrt{Y}}.
]

Joint density of ((Y,\vartheta))
After Step a, ((U_1,U_2)) is uniform on the unit disk (D={(u_1,u_2): u_1^2+u_2^2 \le 1}), with density (f_{U_1,U_2}(u_1,u_2)=\frac{1}{\pi},\mathbf{1}D(u_1,u_2)). Consider the polar-type transform
[
(u_1,u_2) = \big(\sqrt{y}\cos\theta,\ \sqrt{y}\sin\theta\big), \qquad y \in [0,1],\ \theta \in [0,2\pi).
]
The Jacobian determinant is
[
\left|\frac{\partial(u_1,u_2)}{\partial(y,\theta)}\right|
= \left|
\begin{matrix}
\frac{1}{2\sqrt{y}}\cos\theta & -\sqrt{y}\sin\theta \
\frac{1}{2\sqrt{y}}\sin\theta & \ \sqrt{y}\cos\theta
\end{matrix}
\right|
= \tfrac{1}{2}.
]
Hence
[
f{Y,\vartheta}(y,\theta)
= f_{U_1,U_2}\big(\sqrt{y}\cos\theta,\sqrt{y}\sin\theta\big)\cdot \tfrac{1}{2}
= \frac{1}{2\pi},\mathbf{1}{[0,1]}(y),\mathbf{1}{[0,2\pi)}(\theta).
]
Thus (Y \sim \mathrm{U}[0,1]), (\vartheta \sim \mathrm{U}[0,2\pi)), and (Y \perp!!!\perp \vartheta).

2. Distribution of ((X_1,X_2))

Write (U_1=\sqrt{Y}\cos\vartheta), (U_2=\sqrt{Y}\sin\vartheta). Then
[
X_1 = \sqrt{-2\log Y},\cos\vartheta, \qquad
X_2 = \sqrt{-2\log Y},\sin\vartheta.
]
Let (R = \sqrt{-2\log Y}). Since (Y \sim \mathrm{U}(0,1)), for (r>0),
[
\mathbb{P}(R \le r) = \mathbb{P}(-2\log Y \le r^2) = \mathbb{P}\big(Y \ge e^{-r^2/2}\big) = 1 - e^{-r^2/2},
]
so (f_R(r) = r e^{-r^2/2}) for (r>0). Moreover, (R \perp!!!\perp \vartheta) and (\vartheta \sim \mathrm{U}[0,2\pi)).

The mapping ((r,\theta) \mapsto (x_1,x_2)=(r\cos\theta,r\sin\theta)) has Jacobian determinant (r), hence
[
f_{X_1,X_2}(x_1,x_2)
= \frac{f_{R,\vartheta}(r,\theta)}{r}
= \frac{1}{2\pi},e^{-r^2/2}
= \frac{1}{2\pi}\exp!\Big(-\frac{x_1^2+x_2^2}{2}\Big).
]
Therefore ((X_1,X_2) \sim \mathcal{N}_2(0,I_2)), i.e., (X_1,X_2) are independent (\mathcal{N}(0,1)).

3. Potential benefits over the Box–Muller algorithm

No trigonometric evaluations: the polar method avoids (\sin) and (\cos). It uses only one (\log), one (\sqrt{\cdot}), and basic arithmetic per accepted pair.
Despite the rejection step (acceptance probability (\pi/4 \approx 0.785)), it typically runs faster than Box–Muller on many architectures because (\sin) and (\cos) are expensive.
It returns two independent (\mathcal{N}(0,1)) variates per accepted iteration, as Box–Muller does.
It is simple and well-suited to vectorization/SIMD.
Trade-off: the expected number of uniform variates per accepted pair is ((4/\pi)\times 2 = 8/\pi \approx 2.546), versus exactly 2 for Box–Muller; the elimination of trigonometric calls usually compensates for this overhead.

