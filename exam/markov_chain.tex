\documentclass[aspectratio=169]{beamer}
\usepackage{beamerstyle}

\begin{document}

\begin{frame}
	\vspace{2cm}
	\begin{center}
		{\Huge\textbf{\textcolor{copenhagenred}{Markov Chains}}}
		\vspace{1cm}

		\rule{4cm}{3pt}
		\vspace{2cm}
	\end{center}
\end{frame}

\begin{frame}{What is a Markov Chain?}
    \textbf{Definition:} A discrete-time process \textit{Markov Chain} is a sequence of random variables $\{X_t\}_{t \geq 0}$ with the property that, given the present state, the future and past states are independent. Formally,
    \begin{equation*}
        P(X_{t+1} | X_t, X_{t-1}, \ldots, X_0) = P(X_{t+1} | X_t).
    \end{equation*}
    The Markov Chain is \textbf{time-homogeneous} if the transition probabilities do not depend on time $t$:
    \begin{equation*}
        \forall n \in N, \quad P(X_t = y | X_{t-1} = x) = P(X_{t+m} = y | X_{t+m-1} = x)
    \end{equation*}
    i.e. transition probabilities do not depend on $t$.
    The \textbf{key idea} is MCMC is to construct a Markov Chains such that $x_t$ converges 
    to a desired distribution $\pi$ as $t \to \infty$ and
    \begin{equation*}
        \frac{1}{n} \sum_{t=1}^n \phi(x_t) \to \mathbb{E}_{x \sim \pi}[\phi(x)] \quad \text{as } n \to \infty.
    \end{equation*}
    What kinds of conditions are required for this to hold?
\end{frame}

\begin{frame}{Invariant / stationary distribution}
    A distribution $\pi$ is called \textbf{invariant} (or \textbf{stationary}) for a Markov Chain with transition kernel $P$ if
    \begin{equation*}
        \pi(y) = \int \pi(x) P(x, y) dx.
    \end{equation*}
    Intuitively, if the chain starts with distribution $\pi$, it remains in distribution $\pi$ at all future times.

    \vspace{0.5cm}
    Time-homogeneous is not needed for invariant distribution. But it is often easier to verify in that case.
\end{frame}

\begin{frame}{Irreducible}
    A Markov Chain is called \textbf{irreducible} if it is possible to get to any state 
    from any state. Formally, for any states $x$ and $y$, there exists an integer
    $0 \leq n < \infty$ such that
    \begin{equation*}
        P^n(x, y) > 0,
    \end{equation*}
    where $P^n(x, y)$ is the $n$-step transition probability from state $x$ to state $y$.
\end{frame}

\begin{frame}{Aperiodicity}
    A Markov Chain is called \textbf{aperiodic} if it does not get trapped in cycles with fixed periods. Ensures actual convergence instead of oscillation.
    Formally, for any state $x$, the greatest common divisor of the set of integers
    \begin{equation*}
        \{ n \geq 1 : P^n(x, x) > 0 \}
    \end{equation*}
    is 1.
    \textbf{Note:} If all states have a non-zero probability of remaining in the same state, the chain is aperiodic.
\end{frame}

\begin{frame}{Positive recurrence}
    A Markov Chain is called \textbf{positive recurrent} if, starting from any state, the expected return time to that state is finite. Formally, for any state $x$,
    \begin{equation*}
        \mathbb{E}[T_x | X_0 = x] < \infty,
    \end{equation*}
    where $T_x$ is the return time to state $x$.
    \textbf{Note:} Positive recurrence ensures that the chain does not wander off to infinity and has a well-defined long-term behavior.
\end{frame}

\begin{frame}{More on recurrence}
    \begin{itemize}
        \item \textbf{Recurrent}: A Markov Chain is called recurrent if, starting from any state, the probability of returning to that state is 1.
        \item \textbf{Positive recurrence:} A Markov Chain is called positive recurrent if it is recurrent and the expected return time to any state is finite, i.e. the chain returns quickly on average.
        \item \textbf{Transient}:
        \item \textbf{Null recurrence:} A Markov Chain is called null recurrent if it is recurrent but the expected return time to any state is infinite.
    \end{itemize}
\end{frame}

\end{document}